\documentclass[10pt,noamssymb,svgnames]{beamer}
\usetheme{metropolis}

\usepackage{amsmath}
\usepackage{amssymb}
\usepackage{booktabs}
\usepackage{ragged2e}
\usepackage[vlined]{algorithm2e}
\usepackage[absolute,overlay]{textpos}
\usepackage{tikz}
\usetikzlibrary{shapes.geometric,arrows}

\tikzset{onslide/.code args={<#1>#2}{%
  \only<#1>{\pgfkeysalso{#2}} % \pgfkeysalso doesn't change the path
}}
\tikzset{temporal/.code args={<#1>#2#3#4}{%
  \temporal<#1>{\pgfkeysalso{#2}}{\pgfkeysalso{#3}}{\pgfkeysalso{#4}} % \pgfkeysalso doesn't change the path
}}

\usepackage{hyperref}
\hypersetup{colorlinks=true,urlcolor=cyan,linkcolor=}

\definecolor{CyanBlueAzure}{HTML}{4B8BBE} % light blue
\definecolor{LapisLazuli}{HTML}{306998} % dark blue

% BEAMER CONFIGURATION ---------------------------------------------------------
\newcommand<>{\blue}[1]{{\color#2{blue}#1}}
\setbeamerfont{block title}{size=\normalsize}
\setbeamerfont{block body}{size=\scriptsize}
\setbeamertemplate{mini frames}{}

%\setbeamercolor{sectiontitle}{fg=black}
\setbeamercolor{part title}{fg=black}
\setbeamercolor{progress bar}{fg=CyanBlueAzure}
\setbeamercolor{progress bar in head/foot}{fg=LapisLazuli,bg=CyanBlueAzure}
\setbeamercolor{progress bar in section page}{fg=LapisLazuli,bg=CyanBlueAzure}
\setbeamercolor{frametitle}{bg=black!75!white}

% Configure progress bars
\metroset{progressbar=frametitle}
\makeatletter
\setlength{\metropolis@progressonsectionpage@linewidth}{1pt}
\setlength{\metropolis@progressinheadfoot@linewidth}{2pt}
\makeatother

% Blank footnote
\newcommand\blfootnote[1]{%
  \begingroup
  \renewcommand\thefootnote{}\footnote{#1}%
  \addtocounter{footnote}{-1}%
  \endgroup
}

\bibliographystyle{ans}
\setbeamertemplate{bibliography item}{\insertbiblabel}

\newcommand{\highlight}[1]{%
  \colorbox{yellow!20}{$\displaystyle#1$}}

% TIKZ CONFIGURATION -----------------------------------------------------------
\tikzstyle{start} = [rectangle, draw, fill=green!20, rounded corners=3mm,
  centered, minimum height=1em]
\tikzstyle{end} = [rectangle, draw, fill=red!20, rounded corners=3mm,
  centered, minimum height=1em]
\tikzstyle{process} = [rectangle, draw, fill=yellow!20, text width=8em, text
  centered, minimum height=1em]
\tikzstyle{decision} = [diamond, draw, fill=gray!20, text width=5em, text badly centered,
  inner sep=0pt]
\tikzstyle{line} = [draw, -latex', thick]

%%---------------------------------------------------------------------------%%
\title{Introduction to OpenMC}
\author{Technical Meeting on the Development and Application of Open-Source \\ Modelling and Simulation Tools for Nuclear Reactors \\
Milan, Italy \\
June 20th, 2022}

\date{}
\titlegraphic{
  \begin{picture}(0,0)
    \put(325,-220){\makebox(0,0)[rt]{\includegraphics[height=0.75cm]{images/openmc_logo.png}}}
  \end{picture}}

%%---------------------------------------------------------------------------%%
\begin{document}

\maketitle

%%---------------------------------------------------------------------------%%
\begin{frame}{Objectives}
  The overarching objectives of the OpenMC project:
  \begin{itemize}
    \item Open source contribution model, freely available
    \item Extensible for research purposes
    \item Adopt best practices for software development
    \item Ease of installation, minimize third-party dependencies
    \item High performance, scalable on HPC resources
    \item Use best physics models when possible
    \item Fun to use with a thriving user and developer community!
  \end{itemize}
\end{frame}

\begin{frame}{OpenMC: Overview of features}
  \begin{itemize}
    \item \textbf{Modes}: Fixed source, $k$-eigenvalue calculations, stochastic
    volume calculation, geometry plotting
    \item \textbf{Geometry}: Constructive solid geometry, CAD-based,
    unstructured mesh (tallies only)
    \item \textbf{Solvers}: Neutron and photon transport, depletion
    \item \textbf{Data}: Continuous energy or multigroup cross sections,
    multipole for Doppler broadening
  \end{itemize}
\end{frame}

\begin{frame}{What makes OpenMC unique?}
  \begin{itemize}
    \item Programming interfaces (C/C++ and Python)
    \item Nuclear data interfaces and representation
    \item Tally abstractions
    \item Parallel performance
    \item Development workflow and governance
  \end{itemize}
\end{frame}

\begin{frame}{Parallel Performance}
  \begin{columns}
    \column{0.35\textwidth}
    \begin{center}
      \includegraphics[width=1.7in]{images/mcperformance.pdf}
    \end{center}
    \footnotesize{
      \begin{itemize}
      \item ALCF Mira supercomputer
      \item 49,152 nodes, 786,432 cores
      \item 4 hw threads/core = 3,145,728 threads
      \end{itemize}
    }
    \column{0.65\textwidth}
    \includegraphics[width=3.0in]{images/scaling_loglog.pdf}
  \end{columns}
\end{frame}

% \begin{frame}{Example: Advanced Test Reactor}
%   \begin{center}
%     \includegraphics[width=0.7\textwidth]{images/atr.png}
%   \end{center}
% \end{frame}

\begin{frame}{Software Architecture}
  \begin{itemize}
  \item Mixed \textbf{C++} and \textbf{Python} codebase
  \item \textbf{CMake} build system for portability
  \item Distributed-memory parallelism via \textbf{MPI}
  \item Shared-memory parallelism via \textbf{OpenMP}
  \item Version control through \textbf{git}
  \item Code hosting, bug tracking through \href{https://github.com/openmc-dev/openmc}{\textbf{GitHub}}
  \item Regression/unit tests run on \textbf{GitHub Actions} CI platform
  \end{itemize}
\end{frame}

\begin{frame}{Other Capabilities}
  \begin{itemize}
    \item Depletion
    \item Multi-group cross sections
    \item CAD-based geometry
    \item Unstructured meshes
    \item Python C-API
  \end{itemize}
\end{frame}

% \begin{frame}{Current/upcoming developments}
%   \begin{itemize}
%     \item GPU porting (Exascale Computing Project)
%     \item Multiphysics coupling
%     \item Fusion shutdown dose rate (SDR) calculations
%     \item Unstructured mesh support
%     \item Methods to support molten salt reactor design
%   \end{itemize}
% \end{frame}

\begin{frame}{Resources}
  \begin{itemize}
    \item \textbf{Code:} \url{https://github.com/openmc-dev/openmc}
    \item \textbf{Docs:} \url{https://docs.openmc.org}
    \item \textbf{Nuclear Data:} \url{https://openmc.org}
    \item \textbf{Forum:} \url{https://openmc.discourse.group}
    \item \textbf{Examples:} \url{https://github.com/openmc-dev/openmc-notebooks}
  \end{itemize}
\end{frame}

\section{Workshop Logistics}

\begin{frame}{Logistics}
  \begin{itemize}
    \item Instructors: Patrick Shriwise, Jon Shimwell, Ben Forget, and others
    \item We will provide each of you with a unique URL that lets you connect to
    a Jupyter Lab instance running on a cloud server with OpenMC preinstalled
    \begin{itemize}
      \item \textbf{URL}: https://oncore-\#.openmccourse.org
      \item \textbf{Password}: openmc2022
      \item \url{https://tinyurl.com/oncore-22-urls}
    \end{itemize}
    \item Due to time constraints we won't be covering Python basics today, but here is a link to a \href{https://perso.limsi.fr/pointal/_media/python:cours:mementopython3-english.pdf}{``cheat
    sheet''}
    \item Follow along and type the same commands as we go (or not!)
    \item Feel free to ask questions on (either live or on the chat) at any point
  \end{itemize}
\end{frame}



\end{document}
